\documentclass{article}
\usepackage[utf8]{inputenc}

\title{Peer Review of “Top 5 Secure Coding Practices” Presentation}
\author{}
\date{}

\begin{document}

\maketitle

\section{Main Ideas}

\begin{enumerate}
    \item \textbf{Input Validation:} Ensures only properly formatted data is accepted using methods like regular expressions.
    \item \textbf{Output Encoding:} Prevents XSS attacks by encoding special characters in outputs.
    \item \textbf{Authentication and Password Management:} Focuses on strong password policies.
    \item \textbf{Session Management:} Handles session IDs securely and renews them periodically.
    \item \textbf{Access Control:} Uses least privilege principles to regulate access to resources.
    \item \textbf{Error Handling and Logging:} Avoids revealing sensitive information in error messages and logs securely.
    \item \textbf{Data Protection:} Maintains data confidentiality and integrity through encryption.
    \item \textbf{Communication Security and Database Security:} Uses TLS for secure data transmission and safe database operations.
    \item \textbf{File and Memory Management:} Implements safe file uploads and prevents memory vulnerabilities like buffer overflows.
\end{enumerate}

\section{Topics in the Discussion Forum}

The discussion forum encouraged participants to explore various aspects of secure coding practices.

\begin{itemize}
    \item The differences between client-side and server-side validation.
    \item The role of output encoding in preventing vulnerabilities.
    \item Techniques for secure password storage.
    \item Best practices for session management.
\end{itemize}

\section{Critique of the Presentation Slide (30 Points)}

\subsection{Improvement Over the Original (2023 Version)}

The current version is comprehensive. 

\textbf{Recommendation:} Add a comparative chart or a “What’s New” section to highlight changes since 2023.

\subsection{Flow}

The information flows logically from basic to more specific topics.

\textbf{Recommendation:} Use transition slides or a case study to show how practices are connected.

\subsection{Coverage}

All topics are covered with examples but some areas could use more detailed real-world examples.

\textbf{Recommendation:} Add practical use cases to provide a more complete discussion.

\section{Critique of the Discussion Forum Activity (10 Points)}

\subsection{Questions/Comments}

The questions are relevant focusing on key aspects of secure coding.

\textbf{Recommendation:} Include scenario-based or open-ended questions.

\subsection{Major Issues}

The questions address core concerns like XSS prevention.

\textbf{Recommendation:} Add questions about new threats or trends such as AI-related vulnerabilities.

\section{Critique of the Discussion Forum Interaction (10 Points)}

\subsection{Presenter Engagement}

It's hard to assess engagement but the forum structure suggests active participation.

\textbf{Recommendation:} Highlight notable discussions or presenter contributions to show engagement.

\subsection{Further Discussion}

The questions encourage meaningful discussion but could use follow-up prompts.

\section{Recommendations for Improvement}

\subsection{Presentation Content}

Use more visuals like diagrams to explain complex concepts and include real-world case studies.

\subsection{Forum Activity}

Encourage participants to share personal experiences or industry-specific challenges.


\end{document}