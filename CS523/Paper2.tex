\documentclass[12pt]{article}
\usepackage{geometry}
\geometry{a4paper, margin=1in}
\usepackage{amsmath, amssymb}
\usepackage{graphicx}
\usepackage{hyperref}
\usepackage{fancyhdr}
\usepackage{enumitem}
\usepackage{setspace}
\usepackage{hyperref}
\usepackage{xcolor}
 
\title{Large Language Model System Policy}
\author{Abraham J. Reines}
\date{7 July, 2024}
 
\pagestyle{fancy}
\fancyhf{}
\fancyhead[L]{Large Language Model System Policy}
\fancyhead[R]{Abraham J. Reines}
\fancyfoot[C]{\thepage}
 
\begin{document}
 
\maketitle
 
\tableofcontents
 
\newpage
 
\section{Introduction}
 
\subsection{Brief Description of Organization}
 
ByteMeXpert is a freelance software development business specializing in custom bespoke marketing computer program solutions. Our organization values innovation, client satisfaction, as well as conscientious data security. As a growing business, it is imperative to establish a robust policy for usage of Large Language Model (LLM) systems for total compliance with legal standards. Entwined with these facets of ByteMeXpert, is high standards of ethical performance.  Resulting in bonds of trust with our clients, partners and teammates.
 
\section{Policy for Employee Usage of LLM systems}
 
\subsection{Email and Web Usage}
 
\begin{enumerate}
    \item Employees use organizational email accounts for business communications. Personal email accounts will not be used for work related activities.
    \item Internet access is for business purposes exclusively. Limited personal use is permitted provided it does not interfere with work activities.
    \item Employees must refrain from accessing inappropriate, illegal, or non-work-related sites on work machines.
    \item All email and web activity is monitored for compliance with policy.
    \item Use of ByteMeXpert portal is required for all business communications.
    \item Personal information including social security numbers and dates of birth will be subject to rejection over company email systems.
    \item Employees are required to undergo annual training regarding email and web usage policy.
                Phishing exercises will be deployed through company email to test employee’s awareness of avoiding email scams.
               
\end{enumerate}
 
\subsection{Control of Personal Software}
 
\begin{enumerate}
    \item Installation of personal software on organizational assets is prohibited.
    \item All software used must be approved by organization Creator.
    \item Employees are responsible for machines used for work purposes complying with organizational security precautions.
    \item Unauthorized software may be audited/removed by organization Creator.
    \item Quarterly audits will be conducted on company software and networks and violations of policy will result in additional training, suspension or termination.
\end{enumerate}
 
\section{Ethical, Moral, and Legal Implications}
\label{sec:implications}
 
\subsection{Ethical Implications}
\label{subsec:ethical-implications}
 
ethical implications of this policy are integrity/confidentiality of client and organizational data. By restricting email and web usage, sensitive information is not compromised. Monitoring employee activity raises ethical concerns about privacy; it is necessary measure for safeguarding organizations against potential threats. Violations of policy will be carefully measured and weighed based on impact of potential security breaches.
 
\subsection{Moral Implications}
\label{subsec:moral-implications}
 
From a moral standpoint policy promotes responsible use of organizational resources. Employees are expected to use these resources in a manner which aligns with organizations values. prohibition of personal software safeguards integrity of organizational systems.
 
\subsection{Legal Implications}
\label{subsec:legal-implications}
 
Legally policy is compliance with various data protection industry standards. By regulating email/web usage organizations mitigate risk associated with cyber threats. Policy is aimed at protection of organization and employee’s and clients personal information. control of personal software use make sure all software on organizational systems is properly licensed, thereby avoiding potential legal issues related to software piracy and intellectual property rights.
 
\section{Conclusion}
\label{sec:conclusion}
 
In conclusion, policy for employee usage of organizational LLM systems is designed for maintaining a secure, ethical, and legally compliant work environment. It addresses measures to protect organization and employees from potential risks form email, web usage, and personal software. By understanding these guidelines employees contribute to overall integrity and success of ByteMeXpert.

\newpage
\begin{thebibliography}{9}
\bibitem{jmu}
United States Congress. (2008). Ryan Haight Online Pharmacy Consumer Protection Act of 2008. \bibitem{}
Public Law 110-425. Retrieved from \url{https://www.congress.gov/110/plaws/publ425/PLAW-110publ425.pdf}
\bibitem{}
European Union. (2016). General Data Protection Regulation. Official Journal of the European Union, L 119, 1-88. Retrieved from \url{https://eur-lex.europa.eu/eli/reg/2016/679/oj}
\bibitem{}
\url{https://app.originality.ai/share/smyruzxb0wjq8v3c}
\end{thebibliography}

\vfill
\section*{Academic Integrity Pledge}
{\color{red}\textit{“This work complies with JMU honor code. I did not give or receive unauthorized help on this assignment.”}}

\end{document}