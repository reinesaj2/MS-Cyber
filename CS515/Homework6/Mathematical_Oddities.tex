\documentclass[12pt]{article}

% Packages
\usepackage{amsmath}
\usepackage{amssymb}
\usepackage{amsthm}
\usepackage{geometry}
\usepackage{setspace}
\usepackage{xcolor}

% Define theorem, definition, and proof environments
\newtheorem{theorem}{Theorem}
\newtheorem{definition}{Definition}
\newtheorem{lemma}{Lemma}
\newtheorem{corollary}{Corollary}
\newtheorem{proposition}{Proposition}
\newtheorem{example}{Example}
\newtheorem{remark}{Remark}

% Set up the page margins
\geometry{left=0.5in,right=0.5in,top=0.5in,bottom=0.75in}

\begin{document}
\doublespacing

\title{HW-04 Part One}
\author{Abraham J. Reines}
\date{\today}
\maketitle

\section{4.2.22: Parity of \( x^2 + x \) for Odd \( x \)}

\textbf{Objective:} The purpose of this examination is to delve into the arithmetic properties of integers, specifically odd integers, and to ascertain the veracity of the following conjecture:
\[
\text{If \( x \) is an odd integer, then \( x^2 + x \) is even?}
\]

\textbf{Theorem:}

Let \( x \) be an arbitrary odd integer.

\textbf{Proof:}

We begin our proof by using a basic property of integer arithmetic: the product of any two odd integers is odd (Property 3). From this property, it follows \( x^2 \) is an odd integer, assuming \( x \) itself is odd.

Next, we note \( x \) is given as an odd integer for this proof.

Using another integer arithmetic property—namely the sum of two odd integers is even (Property 2)—we can safely conclude \( x^2 + x \) is even.

Thus, we confirm our initial statement is true: for any odd integer \( x \), \( x^2 + x \) is indeed even.

\section{4.2.28: Rational Solutions for \( \frac{ax+b}{cx+d} = 1 \)}

\textbf{Objective:} Given integers \( a, b, c, \) and \( d \) with the constraint \( a \neq c \), must \( x \) be rational? If so, express \( x \) as a ratio of two integers when it satisfies the equation:
\[
\frac{ax+b}{cx+d} = 1
\]

\textbf{Theorem:}

In the context of the equation above, \( x \) must be a ratio of two integers.

\textbf{Proof:}

We start by applying cross-multiplication to the equation, resulting in:
\[
ax + b = cx + d
\]
Further simplification yields:
\[
ax - cx = d - b
\]
\[
x(a - c) = d - b
\]
\[
x = \frac{d - b}{a - c}, \quad a \neq c
\]

Let \( n = d - b \) and \( m = a - c \). Both \( n \) and \( m \) are integers, as the difference between any two integers is also an integer. Additionally, \( m \neq 0 \) because \( a \neq c \).

Therefore, \( x = \frac{n}{m} \) for integers \( n \) and \( m \) where \( m \neq 0 \).

This confirms \( x \) is a ratio of two integers, satisfying our initial objective and proving our theorem.

\textbf{Conclusion:}

Based on the proof, \( x \) is expressible as a ratio of two integers \( \frac{n}{m} \) where \( m \neq 0 \). This makes \( x \) a rational number.

\end{document}