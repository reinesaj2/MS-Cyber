\documentclass[12pt]{article}

% Packages
\usepackage{amsmath}
\usepackage{amssymb}
\usepackage{amsthm}
\usepackage{geometry}
\usepackage{setspace}
\usepackage{xcolor}
\usepackage{enumitem}

% Define theorem, definition, and proof environments
\newtheorem{theorem}{Theorem}
\newtheorem{definition}{Definition}
\newtheorem{lemma}{Lemma}
\newtheorem{corollary}{Corollary}
\newtheorem{proposition}{Proposition}
\newtheorem{example}{Example}
\newtheorem{remark}{Remark}

% Set up the page margins
\geometry{left=0.5in,right=0.5in,top=0.5in,bottom=0.75in}

\begin{document}
\doublespacing % Use this command for double spacing

\title{HW-05 Part TWO}
\author{Abraham J. Reines}
\date{\today}
\maketitle

\section{Preliminaries 5.6.12}

\begin{definition}
A sequence \( s_0, s_1, s_2, \ldots \) \( \{ s_n \} \) is defined as follows:
\[
s_n = \frac{{(-1)^n}}{{n!}}, \quad \text{for all integers } n \geq 0 \hspace{1cm} \textbf{\ldots(1)}
\]
\end{definition}

\section{Main Result}

\begin{theorem}
The sequence \( \{ s_n \} \) satisfies the recurrence relation:
\[
s_k = \frac{{-s_{k-1}}}{{k}}, \quad \text{for all integers } k \geq 1
\]
\end{theorem}

\section{Proof of Theorem}

Let \( k \) be an arbitrary integer such that \( k \geq 1 \).

\begin{enumerate}[label=\arabic*.]
    \item \textbf{Initialization}: We begin by substituting \( n = k - 1 \) into Equation (1) to acquire a base expression for \( s_{k-1} \):
    \[
    s_{k-1} = \frac{{(-1)^{k-1}}}{{(k-1)!}} \hspace{1cm} \textbf{\ldots(2)}
    \]
    
    \item \textbf{Recursion Step}: Next, we substitute \( n = k \) into Equation (1) to acquire an expression for \( s_k \):
    \[
    s_k = \frac{{(-1)^k}}{{k!}}
    \]
    
    \item \textbf{Manipulation}: Utilizing the property \( n! = n \cdot (n-1)! \) and the laws of exponents \( a^m \cdot a^n = a^{m+n} \), we rewrite \( s_k \):
    \[
    s_k = \frac{{(-1)^k}}{{k \cdot (k-1)!}} = \frac{{-1}}{{k}} \cdot \frac{{(-1)^{k-1}}}{{(k-1)!}}
    \]
    
    \item \textbf{Substitution}: Using Equation (2), we rewrite \( s_k \) in terms of \( s_{k-1} \):
    \[
    s_k = \frac{{-1}}{{k}} \cdot s_{k-1}
    \]
\end{enumerate}
Hence, \( s_k = \frac{{-s_{k-1}}}{{k}} \), for all integers \( k \geq 1 \). This confirms the sequence \( s_0, s_1, s_2, \ldots \) \( \{ s_n \} \) satisfies the stated recurrence relation for \( k \geq 1 \), thereby completing the proof.

\section{Remarks}
\begin{remark}
This proof elucidates the mathematical structure underlying the sequence \( s_0, s_1, s_2, \ldots \) \( \{ s_n \} \), providing insights into its behavior as \( n \) varies. Such analyses are crucial in the broader context of discrete mathematics and its applications.
\end{remark}

\section{Preliminaries 5.6.29}

\begin{theorem}
For all integers \( k \ge 1 \), the following identity holds:
\[ F_{k+1}^2 - F_k^2 = F_{k-1} \cdot F_{k+2} \]
\end{theorem}

\begin{proof}
Consider the Fibonacci sequence \( \{F_n\}_{n=0}^\infty \) defined by the recurrence relation \( F_n = F_{n-1} + F_{n-2} \) for \( n \ge 2 \), with initial conditions \( F_0 = 0 \) and \( F_1 = 1 \).

We will prove the given identity by expanding the left-hand side using the algebraic identity \( x^2 - y^2 = (x - y)(x + y) \):

\[ F_{k+1}^2 - F_k^2 = (F_{k+1} + F_k)(F_{k+1} - F_k) \tag{1} \]

From the recurrence relation of the Fibonacci sequence, we have:

\[ F_{k+1} = F_k + F_{k-1} \tag{2} \]

and

\[ F_{k+2} = F_{k+1} + F_k \tag{3} \]

Substituting the value from equation (2) into the expression \( F_{k+1} - F_k \), we obtain:

\[ F_{k+1} - F_k = F_k + F_{k-1} - F_k = F_{k-1} \tag{4} \]

--

Equation \( (4) \) is derived from the fundamental recurrence relation which defines the Fibonacci sequence:

\[ F_{n} = F_{n-1} + F_{n-2} \]

This relation tells us each term in the sequence is the sum of the two preceding terms, starting with \( F_0 = 0 \) and \( F_1 = 1 \).

For equation \( (4) \), we focus on the terms \( F_{k+1} \) and \( F_k \). The recurrence relation for \( F_{k+1} \) is:

\[ F_{k+1} = F_k + F_{k-1} \]

Now, to express \( F_{k+1} - F_k \), we simply subtract \( F_k \) from both sides of this equation:

\[
\begin{aligned}
F_{k+1} - F_k &= (F_k + F_{k-1}) - F_k \\
&= F_k - F_k + F_{k-1} \\
&= 0 + F_{k-1} \\
&= F_{k-1}
\end{aligned}
\]

The result of this manipulation is the difference between the consecutive Fibonacci numbers \( F_{k+1} \) and \( F_k \) is the prior Fibonacci number \( F_{k-1} \). This is what equation \( (4) \) represents:

\[ F_{k+1} - F_k = F_{k-1} \]

--

Substituting the expressions from equations (3) and (4) into equation (1), we get:

\[ (F_{k+1} + F_k)(F_{k+1} - F_k) = F_{k-1} \cdot F_{k+2} \]

This simplifies to the identity we set out to prove:

\[ F_{k+1}^2 - F_k^2 = F_{k-1} \cdot F_{k+2} \]

Therefore, by direct application of the properties of the Fibonacci sequence and algebraic manipulation, the theorem is proven without the need for mathematical induction.
\end{proof}

\end{document}