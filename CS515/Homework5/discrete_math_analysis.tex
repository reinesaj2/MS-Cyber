\documentclass[12pt]{article}

% Packages
\usepackage{amsmath}
\usepackage{amssymb}
\usepackage{amsthm}
\usepackage{geometry}
\usepackage{setspace}
\usepackage{xcolor}

% Define theorem, definition, and proof environments
\newtheorem{theorem}{Theorem}
\newtheorem{definition}{Definition}
\newtheorem{lemma}{Lemma}
\newtheorem{corollary}{Corollary}
\newtheorem{proposition}{Proposition}
\newtheorem{example}{Example}
\newtheorem{remark}{Remark}

% Set up the page margins
\geometry{left=0.5in,right=0.5in,top=0.5in,bottom=0.75in}

\begin{document}
\doublespacing

\title{HW-03 Part TWO}
\author{Abraham J. Reines}
\date{\today}
\maketitle

\section{3.3.50}
\subsection{Part a: Evaluating the Proposition}
We are tasked with evaluating the veracity of the proposition:
\[
\text{``For every object \( x \), there exists an object \( y \) such that if \( x \neq y \), then \( x \) and \( y \) have different colors.''}
\]
Given the objects are confined to the colors blue, grey, or black, and for each object, there are at least two objects of a different color, the proposition holds as \textbf{True}.

\subsection{Part b: Logical Formalism}
To formally represent the statement in question, we can employ predicate logic. The logical expression becomes:
\[
\forall x \left( \exists y \left( x \neq y \rightarrow \sim (\text{TheSameColor}(x, y)) \right) \right)
\]
Here, \(\text{TheSameColor}(x, y)\) denotes ``\( x \) has the same color as \( y \)''.

\subsection{Part c: Negation in Logical Notation}

\begin{enumerate}
    \item[  ] The objective is to negate the logical statement initially presented in Part b. Starting with the negation of the original logical expression, we have:
    
    \begin{align*}
        \sim (\forall x (\exists y (x \neq y \rightarrow \sim \text{TheSameColor}(x,y)))) &\equiv \exists x (\sim (\exists y (x \neq y \rightarrow \sim \text{TheSameColor}(x,y)))) && {\bf{...(1)}} \\[2em]
        &\equiv \exists x (\forall y (\sim (x \neq y \rightarrow \sim \text{TheSameColor}(x,y)))) &&  {\bf{...(2)}} \\[2em]
        &\equiv \exists x (\forall y (x \neq y \land \sim (\sim \text{TheSameColor}(x,y)))) &&  {\bf{...(3)}} \\[2em]
        &\equiv \exists x (\forall y (\sim (x \neq y \land \text{TheSameColor}(x,y)))) &&  {\bf{...(4)}}
    \end{align*}
    
    We see  {\bf{(1) \& (2)}} are Negation statements. {\bf{(3)}} \( \sim (P \rightarrow Q) \equiv P \land \sim Q\). Finally, {\bf{(4)}} is Negation Law twice. Therefore, the negation of the statement in logical form is \(\exists x (\forall y (x \neq y \land \text{TheSameColor}(x,y)))\).
\end{enumerate}

\section{Logical Argument and Reordered Premises 3.4.32}

The original argument presents the following premises:

\begin{enumerate}
    \item When I work a logic example without grumbling, you may be sure it is one I understood.
    \item The arguments in these examples are not arranged in regular order like the ones I am used to.
    \item No easy example makes my headache.
    \item I can't understand examples if the arguments are not arranged in regular order like the ones I am used to.
    \item I never grumble at an example unless it gives me a headache.
\end{enumerate}

The conclusion drawn is: ``These examples are not easy.''

\subsection{Objective}

The objective is to reorder the premises in the above argument to demonstrate the conclusion is valid.

\subsection{Rewriting the Premises}

We rewrite the premises in an if-then form:

\begin{enumerate}
    \item If I can't understand any logic example, I grumble.
    \item If an argument belongs to these examples, then the arguments are not arranged in regular order like the ones I am used to.
    \item If I get a headache, then the example is not easy.
    \item If the arguments are not arranged in regular order like the ones I am used to, then I can't understand.
    \item If I grumble at any example, then it gives me a headache.
\end{enumerate}

\subsection{Reordered Premises}

Upon reordering, the premises are as follows:

\begin{enumerate}
    \setcounter{enumi}{1}
    \item If an argument belongs to these examples, then the arguments are not arranged in regular order like the ones I am used to.
    \setcounter{enumi}{3}
    \item If the arguments are not arranged in regular order like the ones I am used to, then I can't understand.
    \setcounter{enumi}{0}
    \item If I can't understand any logic example, I grumble.
    \setcounter{enumi}{4}
    \item If I grumble at any example, then it gives me a headache.
    \setcounter{enumi}{2}
    \item If I get a headache, then the example is not easy.
\end{enumerate}

\subsection{Conclusion}

By logically chaining premises 2 and 3, we arrive at the same conclusion: ``These examples are not easy.''

Therefore, the reordered premises validate the conclusion.

\end{document}
