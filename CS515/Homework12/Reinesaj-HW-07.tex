\documentclass[12pt]{article}

% Packages
\usepackage{amsmath}
\usepackage{amssymb}
\usepackage{amsthm}
\usepackage{geometry}
\usepackage{setspace}
\usepackage{xcolor}
\usepackage{enumitem}
\usepackage{graphicx}
\usepackage{listings}
\usepackage{xcolor}
\usepackage{changepage}
\usepackage{endnotes}

% Custom colors for code highlighting
\definecolor{codegreen}{rgb}{0,0.6,0}
\definecolor{codegray}{rgb}{0.5,0.5,0.5}
\definecolor{codepurple}{rgb}{0.58,0,0.82}
\definecolor{backcolour}{rgb}{0.95,0.95,0.92}

% Style for Python code
\lstset{
    language=Python,
    backgroundcolor=\color{backcolour},
    commentstyle=\color{codegreen},
    keywordstyle=\color{magenta},
    numberstyle=\tiny\color{codegray},
    stringstyle=\color{codepurple},
    basicstyle=\ttfamily\small,
    breakatwhitespace=false,
    breaklines=true,
    captionpos=b,
    keepspaces=true,
    numbers=left,
    numbersep=5pt,
    showspaces=false,
    showstringspaces=false,
    showtabs=false,
    tabsize=2
}

% Define theorem, definition, and proof environments
\newtheorem{theorem}{Theorem}
\newtheorem{definition}{Definition}
\newtheorem{lemma}{Lemma}
\newtheorem{corollary}{Corollary}
\newtheorem{proposition}{Proposition}
\newtheorem{example}{Example}
\newtheorem{remark}{Remark}

% Set up the page margins
\geometry{left=0.5in,right=0.5in,top=0.5in,bottom=0.75in}

\begin{document}
\doublespacing % Use this command for double spacing

\title{HW-07}
\author{Abraham J. Reines}
\date{\today}
\maketitle

\section*{7.1.49}

\begin{theorem}
Let \( F: X \to Y \) be a function, and let \( C \) be a subset of \( Y \). Then the following relation holds:
\[ F(F^{-1}(C)) \subseteq C \]
\end{theorem}

\subsection*{Definitions:}
\begin{itemize}
    \item Let \( F: X \rightarrow Y \) be a function.
    \item Define the inverse image of \( C \) under \( F \), \( F^{-1}(C) = \{ x \in X \mid F(x) \in C \} \).
\end{itemize}

\begin{proof}
\subsection*{ \( F(F^{-1}(C)) \subseteq C \):}
\begin{enumerate}
    \item 
    \[
    \text{Let } y \in F(F^{-1}(C)).
    \]
    \textit{(Start with an arbitrary element \( y \) in \( F(F^{-1}(C)) \)}
    
    \item 
    \[
    \text{Then } \exists x \in X \text{ such that } x \in F^{-1}(C) \text{ and } F(x) = y.
    \]
    \textit{(This follows from the definition of \( F(F^{-1}(C)) \), which includes all \( y \in Y \) such that \( y = F(x) \) for some \( x \in F^{-1}(C) \))}
  
    \item 
    \[
    \text{Since } x \in F^{-1}(C), \text{ then } F(x) \in C \text{ and } y \in C.
    \]
    \textit{(By definition of the inverse image, \( F^{-1}(C) \))}
    
    \item 
    \[
    \therefore y = F(x) \in C.
    \]
    \textit{(From the previous step, since \( F(x) = y \), it logically follows \( y \) is an element of \( C \))}
    
    \item 
    \[
    \text{Hence, } F(F^{-1}(C)) \subseteq C.
    \]
    \textit{(This conclusion is drawn by showing every \( y \), which is a result of the function \( F \) applied to any element in \( F^{-1}(C) \), is contained within \( C \), thereby establishing the subset relationship)} 
\end{enumerate}

\subsection*{Conclusion:}
This proof meticulously demonstrates \( F(F^{-1}(C)) \) is a subset of \( C \) for a function \( F: X \to Y \) and a subset \( C \subseteq Y \). The argument is anchored in the fundamental principles of function composition and inverse images in set theory, elucidating these essential concepts in the realm of discrete mathematics.

The last step concludes since every element \( y \) of \( F(F^{-1}(C)) \) is shown to be in \( C \), it follows the entire set \( F(F^{-1}(C)) \) is contained within \( C \). In set-theoretic terms, this means \( F(F^{-1}(C)) \subseteq C \). This is a subset relationship, where every element of the first set is also an element of the second set.

\end{proof}

\section*{7.2.18}
Consider the function \( f \) defined by \( f(x) = \frac{x+1}{x-1} \), where \( x \in \mathbb{R} \setminus \{1\} \). This section aims to determine whether \( f \) is injective and to derive its inverse function, if it exists.

\section*{Preliminaries}
\begin{definition}
A function \( f: A \to B \) is \textit{injective} (or one-to-one) if different elements in \( A \) map to different elements in \( B \), formally \( \forall a, b \in A, f(a) = f(b) \implies a = b \).
\end{definition}

\begin{theorem}
The function \( f(x) = \frac{x+1}{x-1} \) is injective.
\end{theorem}

\begin{proof}
Assume \( f(a) = f(b) \) for some \( a, b \in \mathbb{R} \setminus \{1\} \). Then
\begin{align*}
    \frac{a+1}{a-1} &= \frac{b+1}{b-1} \\
    (a+1)(b-1) &= (b+1)(a-1) \\
    ab - a + b - 1 &= ab + a - b - 1 \\
    -a + b &= a - b \\
    2b &= 2a \\
    a &= b.
\end{align*}
     Therefore,  \( f \) is injective.
\end{proof}

\subsection*{Inverse:}
\begin{enumerate}
    \item Let \( y = f(x) = \frac{x + 1}{x - 1} \).
    \item To find \( f^{-1}(x) \), solve the equation \( y = \frac{x + 1}{x - 1} \) for \( x \) in terms of \( y \).
    \item Cross-multiplying gives:
    \[ y(x - 1) = x + 1 \]
    \item Distributing and rearranging:
    \[ yx - y = x + 1 \]
    \[ yx - x = y + 1 \]
    \[ x(y - 1) = y + 1 \]
    \[ x = \frac{y + 1}{y - 1} \]
    \item Thus, the inverse function is:
    \[ f^{-1}(y) = \frac{y + 1}{y - 1} \] when \( y \neq 1 \).
\end{enumerate}

\subsection*{Conclusion}
The function \( f: \mathbb{R} \setminus \{1\} \to \mathbb{R} \), given by \( f(x) = \frac{x + 1}{x - 1} \), is shown to be a one-to-one correspondence within its domain. This characterization is affirmed through a demonstration where \( f(a) = f(b) \) for any \( a, b \) in the domain of \( f \), excluding the point \( x = 1 \), compellingly implies \( a = b \). The function's undefined status at \( x = 1 \) does not impinge on its injectivity across the remainder of the real number line. The exclusion of \( x = 1 \) from the domain is pivotal for maintaining the one-to-one nature of the function.

Furthermore, the one-to-one nature of \( f \) within its domain permits the determination of its inverse function. The inverse, denoted as \( f^{-1} \), can be derived by solving the equation \( y = \frac{x + 1}{x - 1} \) for \( x \) in terms of \( y \). This procedure yields \( f^{-1}(y) = \frac{y + 1}{y - 1} \), which is defined for \( y \neq 1 \). Just as \( f \) is undefined at \( x = 1 \), its inverse is undefined at the corresponding \( y \) value that maps to \( x = 1 \) under \( f \), reinforcing the symmetry in the behavior of a function and its inverse.

\end{document}