\documentclass[12pt]{article}

% Packages
\usepackage{amsmath}
\usepackage{amssymb}
\usepackage{amsthm}
\usepackage{geometry}
\usepackage{setspace}
\usepackage{xcolor}
\usepackage{enumitem}

% Define theorem, definition, and proof environments
\newtheorem{theorem}{Theorem}
\newtheorem{definition}{Definition}
\newtheorem{lemma}{Lemma}
\newtheorem{corollary}{Corollary}
\newtheorem{proposition}{Proposition}
\newtheorem{example}{Example}
\newtheorem{remark}{Remark}

% Set up the page margins
\geometry{left=0.5in,right=0.5in,top=0.5in,bottom=0.75in}

\begin{document}
\doublespacing % Use this command for double spacing
%\setstretch{3} % Use this command for triple spacing

\title{HW-05 Part ONE}
\author{Abraham J. Reines}
\date{\today}
\maketitle

\section*{Statement 5.1.39}
We are given the summation
\[
\sum_{m=1}^{n+1} m(m+1)
\]
and tasked with rewriting this expression by separating off the final term. The focus is to employ the recursive definition of summation to accomplish this.

\section{Recursive Definition of Summation}
The recursive definition of summation can be mathematically expressed as:

\begin{align*}
    1. \quad \sum_{k=m}^{m} a_k &= a_m, \\
    2. \quad \sum_{k=m}^{n} a_k &= \sum_{k=m}^{n-1} a_k + a_n
\end{align*}

This holds for all integers \( n > m \).

\newpage
  \section{Separating the Final Term}
Using the recursive definition of summation, we can rewrite the original summation as:

\[
\sum_{m=1}^{n+1} m(m+1) = \left( \sum_{m=1}^{n} m(m+1) \right) + [(n+1)((n+1)+1)]
\]

Upon simplification, this becomes:

\[
\sum_{m=1}^{n+1} m(m+1) = \left( \sum_{m=1}^{n} m(m+1) \right) + [(n+1)(n+1+1)]
\]

Further simplifying, we get:

\[
\sum_{m=1}^{n+1} m(m+1) = \left( \sum_{m=1}^{n} m(m+1) \right) + [(n+1)(n+2)]
\]

\section{Conclusion}
The original summation \(\sum_{m=1}^{n+1} m(m+1)\) can be rewritten by separating off the final term \( (n+1)(n+2) \), while the remaining part of the sum goes from \( m = 1 \) to \( n \). This separated form adheres to the recursive definition of summation.

\section*{Problem Statement 5.2.16}

Prove the following statement by mathematical induction:

For all integers \( n \geq 2 \),

\[
(1-\frac{1}{2^2})(1-\frac{1}{3^2}) \ldots (1-\frac{1}{n^2}) = \frac{n+1}{2n}
\]

\newpage
  \section{Base Case: \( n = 2 \)}

First, let's verify the base case, where \( n = 2 \):

\[
\text{Left-hand side} = (1-\frac{1}{2^2}) = \frac{3}{4}
\]
\[
\text{Right-hand side} = \frac{2+1}{2 \times 2} = \frac{3}{4}
\]

Both sides are equal, so the base case holds.

\section{Inductive Step}

Assume \( k \) is an integer \(  \geq 0 \) such that  \( P(k) \) is true for some arbitrary \( n = k \):

\[
(1-\frac{1}{2^2})(1-\frac{1}{3^2}) \ldots (1-\frac{1}{k^2}) = \frac{k+1}{2k}
\]

We need to show \( P(k+1) \) is true for \( n = k+1 \):

\[
(1-\frac{1}{2^2})(1-\frac{1}{3^2}) \ldots (1-\frac{1}{(k+1)^2}) = \frac{(k+1)+1}{2(k+1)}
\]

\subsection{Proving the Inductive Step}

Starting with the left-hand side of the expression for \( n = k+1 \):

\[
\left( (1-\frac{1}{2^2})(1-\frac{1}{3^2}) \ldots (1-\frac{1}{k^2}) \right) (1-\frac{1}{(k+1)^2})
\]

Substitute the inductive hypothesis \( \frac{k+1}{2k} \):

\[
= \frac{k+1}{2k} \times (1 - \frac{1}{(k+1)^2})
\]
\[
= \frac{k+1}{2k} \times \frac{(k+1)^2 - 1}{(k+1)^2}
\]
\[
= \frac{k+1}{2k} \times \frac{k^2 + 2k}{(k+1)^2}
\]
\[
= \frac{(k+1) \times (k^2 + 2k)}{2k \times (k+1)^2}
\]
\[
= \frac{(k+1) \times k \times (k+2)}{2k \times (k+1)^2}
\]
\[
= \frac{(k+2)}{2(k+1)}
\]

This proves the left-hand side equals the right-hand side for \( n = k+1 \):

\[
L.H.S = \frac{(k+1)+1}{2(k+1)} = \frac{k+2}{2(k+1)} = R.H.S
\]

\section{Conclusion}

By mathematical induction, the statement

\[
(1-\frac{1}{2^2})(1-\frac{1}{3^2}) \ldots (1-\frac{1}{n^2}) = \frac{n+1}{2n}
\]

is true for all integers \( n \geq 2 \).

\end{document}
